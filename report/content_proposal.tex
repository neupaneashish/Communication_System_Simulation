\section{What we plan to do}
Our group plans to explore the mimicking of long-exposure times through the post-processing of video.
The goal of this project is to see how much the quality of images can be improved through treating
video as a series of images and using processing to mimic long exposure times. We will compare the
results on 1) cheap cameras 2) a DSLR. Our evaluation will include the following:

\begin{itemize}
\item Low-light performance
\item How well noise can be compensated for or filtered out
\item Situations where the technique is not well suited (e.g. something like median filtering in a
scene with trees/leaves that move)
\item What range of exposure times are practical?
\item Does FPS have an effect on the quality of results?
\end{itemize}

\section{Intermediate Deliverables}

\begin{itemize}
\item Figure out what algorithms to use; e.g. averaging of many frames vs. `smooth waterfall' effect, or
  others

\item Figure out evaluation techniques and metrics

\item Have a database of videos on all test cameras, as well as pictures with DSLR with long exposure time

\item Technique for converting video format into series of still images
\end{itemize}

\section{Preliminary Paper list}

This paper might be useful for a lit review on how frame averaging works, especially with motion in
the video, which may be useful for stabilization

\cite{motion_comp_averaging}

Adaptive hybrid mean and median filtering of high-ISO long-exposure sensor noise for
digital photography
\cite{hyrbid_mean_median_filtering}

Noise reduction in high dynamic range imaging
\cite{noise_HDR}

\section{Feedback}
This sounds cool, and I'm looking forward to seeing some of your results.

It sounds like the first part of your project is really about noise reduction.  Given that, I'd like
to see you do a bit of theoretical examination of the noise sources, and make some predictions about
how many pictures you need to achieve parity with a single long exposure, etc.  I blew through this
way too fast in class, so I'd be happy to work with you in more detail on this.

In general, a single exposure of duration T will have less noise than a combination of N shots each
with duration T/N, so that by itself isn't super exciting.  Given this, there are a few reasons
you'd use multiple exposures:
1. There is motion of the camera (e.g., handshake) or the scene, and you don't want motion blur
2. You're averaging frames to reduce noise, not adding them to make up a long exposure
3. You want a long exposure (e.g., to get motion blur), but that would saturate the sensor.  The
traditional solution here would be to use a neutral-density filter (ND filter) to darken the image,
allowing a longer exposure.

It wasn't completely clear to me from your proposal which of these you're considering.  Are you
going to try and handle alignment to deal with motion blur, or are you just looking at stationary
sequences?

For converting video to a sequence images, check out ffmpeg and libav (they're two forks of the same
open-source project, and either should work).  MATLAB and OpenCV (which also has Python bindings
which integrate nicely with NumPy) also have mechanisms for reading video files as a series of
frames.

\section{Updated Plans}
Lit-review for filtering out moving objects
\begin{itemize}
\item Median Filterting
\item Better techniques
\item Reducing noise
\item Reducing bias (i.e. multimodal dist around true color and a secondary color)
\end{itemize}

This uses Kalman Filtering \cite{moving_object_removal}


